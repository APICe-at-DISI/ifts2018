%\documentclass[12pt,handout]{beamer}
\documentclass[presentation]{beamer}
\usepackage{oop-slides-pianini}
\setbeamertemplate{bibliography item}[text]

\title[Terminale e FS]{01 \\ Terminale e file system}

\author[Pianini]{Danilo Pianini}

\institute[Università di Bologna]
{IFTS \\\textsc{Alma Mater Studiorum}---Università di Bologna, Cesena}

\date[\today]{\today}


\begin{document}

\frame[label=coverpage]{\titlepage}

\section{Il file system}
\begin{frame}{Elementi base del file system}
  \begin{itemize}
    \item I sistemi operativi odierni consentono di memorizzare permanentemente le informazioni su supporti di memorizzazione di massa (dischi magnetici, dispositivi a stato solido), unità ottiche (CD, DVD, Blu-Ray), memory stick, ecc...
    \item Le informazioni su questi supporti sono organizzate in file e cartelle:
      \begin{itemize}
        \item i file contengono le informazioni
        \item le cartelle sono contenitori, all'interno contengono i file ed altre cartelle
      \end{itemize}
    \item La cartella più esterna, che contiene tutte le altre, è detta root. Essa rappresenta il livello gerarchico più alto del file system
      \begin{itemize}
        \item In *nix (Linux, MacOS, BSD, Solaris...), vi è una unica radice, ossia \texttt{/}
        \item In Windows, ciascun file system ha come root una lettera di unità (e.g. \texttt{C:}, \texttt{D:})
      \end{itemize}
    \item La stringa che descrive un intero percorso dalla root fino ad un elemento del file system prende il nome di \emph{percorso} (e.g. \texttt{C:\textbackslash{}Windows\textbackslash{}win32.dll}, \texttt{/home/user/frameworkFS.jar})
  \end{itemize}
\end{frame}

\begin{frame}[fragile]{Manipolare il file system}
  L'utente può osservare e manipolare il file system:
  \begin{itemize}
    \item sapere quali files e cartelle contiene una cartella
    \item creare nuovi files e cartelle
    \item spostare file e cartelle dentro altre cartelle
    \item rinominare files e cartelle
    \item eliminare files e cartelle
  \end{itemize}
  Il software che consente di osservare e manipolare il file system prende il nome di \alert{file manager}.
  \begin{itemize}
    \item Su Windows, esso è ``Esplora risorse'' (\texttt{explorer.exe})
    \item Su MacOS, il principale è ``Finder''
    \item Su Linux (e Android) ne esistono diversi (Nautilus, Dolphin, Thunar, Astro...)
  \end{itemize}
\end{frame}




\section{Richiamo: Interprete Comandi}
\fr{Interprete Comandi}{
  \bl{}{
    Programma che permette di interagire con il S.O. mediante comandi impartiti in modalità testuale (non grafica), via linea di comando
  }
% 	\bl{Scripting}{
% 		L'interprete comandi può avere un linguaggio associato con cui è possibile scrivere script
% %		\iz{
% %			\item Utili (principalmente) per automatizzare task di diversa natura
% %		}
% 	}
% 	\bl{Vari tipi di comandi}{
% 		\iz{
% 			\item Navigazione file system
% 			\item Interazione con il file system
% 			\item Esecuzione di programmi da riga di comando
% 			\item ...
% 		}
% 	}
  \begin{itemize}
    \item Nell'antichità (in termini informatici) le interfacce grafiche erano sostanzialmente inesistenti, e l'interazione con i calcolatori avveniva di norma tramite interfaccia testuale
    \item Tutt'oggi, le interfacce testuali sono utilizzate:
    \begin{itemize}
      \item per automatizzare le operazioni
      \item per velocizzare le operazioni (scrivere un comando è spesso molto più veloce di andare a fare click col mouse in giro per lo schermo)
      \item per fare operazioni complesse con pochi semplici comandi
      \item non tutti i software sono dotati di interfaccia grafica
      \item alcune opzioni di configurazione del sistema operativo restano accessibili solo via linea di comando
      \begin{itemize}
        \item (anche su Windows: ad esempio i comandi per associare le estensioni ad un eseguibile)
      \end{itemize}
    \end{itemize}
  \end{itemize}
  \bl{}{
    Lo vedrete in maniera esaustiva nel corso di Sistemi Operativi...
  }
}

\fr{Sistemi *nix (Linux, MacOS X, FreeBSD, Minix...)}{
        \bl{Nei sistemi UNIX esistono vari tipi di interpreti, chiamati shell}{
                Alcuni esempi
                \iz{
                        \item Bourne shell (sh)
                                \iz{ \item Prima shell sviluppata per UNIX (1977)}
                        \item C-Shell (csh)
                                \iz{ \item Sviluppata da Bill Joy per BSD}
                        \item Bourne Again Shell (bash)
                                \iz{ \item Parte del progetto GNU, è un super set di Bourne shell}
                        \item ...
                }
        }
        \bl{Per una panoramica completa delle differenze}{
                \textcolor{blue}{\url{http://www.faqs.org/faqs/unix-faq/shell/shell-differences/}}
        }
}

\fr{Sistemi Windows}{
  \bl{}{
    L'interprete comandi è rappresentato dal programma \texttt{cmd.exe} in \texttt{C:$\backslash$Windows$\backslash$System32$\backslash$cmd.exe}
    \iz{
      \item Eredita in realtà sintassi e funzionalità della maggior parte dei comandi del vecchio MSDOS
      \item Versioni recenti hanno introdotto PowerShell, basato su .NET e C\#
      \item Windows 10 ha introdotto il supporto a bash tramite Linux Subsystem for Linux
    }
  }
  \fg{height=0.4\textheight}{img/prompt.png}
}

\begin{frame}[fragile]{Aprire un terminale in laboratorio}
  \begin{itemize}
    \item In laboratorio, troverete il terminale (prompt dei comandi) clickando su Start $\Rightarrow$ Programmi $\Rightarrow$ Accessori $\Rightarrow$ Prompt dei comandi
    \item Metodo più rapido: Start  $\Rightarrow$ Nella barra di ricerca, digitare \texttt{cmd} $\Rightarrow$ clickare su \texttt{cmd.exe}
  \end{itemize}
\end{frame}

\begin{frame}[fragile]{File system e terminale: cheat sheet}
\label{slide:commands}
  \begin{center}
    \begin{tabular}{| l | c | c |}
      \hline
      \textbf{Operazione} & \textbf{Comando Unix} & \textbf{Comando Win} \\ \hline
      \scriptsize{}Visualizzare la directory corrente & \texttt{pwd} & \texttt{echo \%cd\%}  \\ \hline
      \scriptsize{}Eliminare il file \texttt{f} (non va con le cartelle!) & \texttt{rm f} & \texttt{del f} \\ \hline
      \scriptsize{}Eliminare la directory \texttt{nd} & \texttt{rm -r nd} & \texttt{rd nd} \\ \hline
      \scriptsize{}Contenuto della directory corrente & \texttt{ls -alh} & \texttt{dir} \\ \hline
%       \scriptsize{}Avviare un eseguibile di nome \texttt{pippo} & \texttt{./p} & \texttt{.\textbackslash{}p} \\ \hline
      \scriptsize{}Cambiare unità disco (passare a D:) & -- & \texttt{D:} \\ \hline
      \scriptsize{}Passare alla directory \texttt{nd} & \texttt{cd nd} & \texttt{cd nd} \\ \hline
      \scriptsize{}Passare alla directory di livello superiore & \texttt{cd ..} & \texttt{cd..} \\ \hline
      \scriptsize{}Spostare (rinominare) un file \texttt{f1} in \texttt{f2} & \texttt{mv f1 f2} & \texttt{move f1 f2} \\ \hline
      \scriptsize{}Copiare il file \texttt{f} in \texttt{fc} & \texttt{cp f fc} & \texttt{copy f fc} \\ \hline
      \scriptsize{}Creare la directory \texttt{d} & \texttt{mkdir d} & \texttt{md d} \\ \hline
    \end{tabular}
  \end{center}
  Eseguire delle prove ed esser certi di aver compreso come utilizzare ogni comando. Per \emph{cominciare} l'esame, in particolare, dovrete usare il comando \texttt{cd}: siate certi di aver capito cosa fa!
\end{frame}

\begin{frame}[fragile]{Uso intelligente del terminale}
  \begin{block}{Autocompletamento}
    \scriptsize{}
    Sia *nix che Windows offrono la possibilità di effettuare autocompletamento, ossia chiedere al sistema di provare a completare un comando. Per farlo si utilizza il tasto ``tab'' (quello con due frecce orientate in maniera opposta, sopra il lucchetto).
  \end{block}
  \begin{block}{Memoria dei comandi precendenti}
    \scriptsize{}
    Sia *nix che Windows offrono la possibilità di richiamare rapidamente i comandi inviati precedentemente premendo il tasto ``freccia su''. I sistemi *nix supportano anche il lancio di comandi eseguiti in sessioni precedenti (non perde memoria col riavvio del terminale). 
  \end{block}
  \begin{block}{Interruzione di un programma}
    \scriptsize{}
    È possibile interrompere forzatamente un programma (ad esempio perché inloopato). Per farlo, sia su Windows che in *nix, si prema ctrl+c.
  \end{block}
  \begin{block}{Ricerca nella storia dei comandi precedenti}
    \scriptsize{}
    Premendo ctrl+r seguito da un testo da cercare, i sistemi *nix supportano la ricerca all'interno dei comandi lanciati recentemente, anche in sessioni utente precedenti. Non disponibile su Windows.
  \end{block}
\end{frame}

\end{document}


%====================
%Java come Linguaggio Object-Oriented
%====================
\fr{Java come Linguaggio Object-Oriented}{
	\bl{Un po' di storia}{Java nasce nei laboratori della Sun microsystem per l'embedded networking computing agli inizi degli anni '90
		\iz{
			\item Handheld home-entertainment controller: primo \textit{testbed} per il linguaggio, pensato per la TV via cavo
				\iz{\item Abbandonato perché troppo complesso per quello specifico contesto}
			\item Poi applicato - inizialmente - in ambito web (Netscape browser, 1995)
			\item Oggi uno dei linguaggi più usati al mondo
			\item Per maggiori dettagli sulla storia del linguaggio \cite{java-history-wiki,java-early}
		}
	}
	
	\bl{(Almost) Everything is an Object}{
		\iz{
			\item Tranne i tipi primitivi base (\cil{int}, \cil{float}, ...)...
			\item .. esistono solo \alert{classi} (a livello statico) e \alert{oggetti} (a runtime)
%				\iz{
%					\item Anche il \textit{main} è definito dentro a una classe!
%				}
		}
	}
}

